% Adapted from http://www.cv-templates.info/2009/10/classicthesis-currvita-latex-cv-template/

\documentclass{scrartcl}        % classe article di KOMA

\makeatletter
%\let\saved@bibitem\@bibitem
\makeatother

\reversemarginpar
\newcommand{\MarginDate}[1]{\marginpar{\raggedleft\itshape\small#1}}
\usepackage[LabelsAligned]{currvita}    % un buon pacchetto per CV
%\usepackage[eulermath,nochapters]{classicthesis} % stile ClassicThesis
\usepackage[nochapters]{classicthesis} % stile ClassicThesis
\usepackage{url}        % per gli indirizzi Internet
%\usepackage{cite}
\usepackage{natbib}
\usepackage{bibentry}
\usepackage{fullpage}
\renewcommand{\cvheadingfont}{\LARGE\color{Maroon}}
\renewcommand{\cvlistheadingfont}{\large}
\renewcommand{\cvlabelfont}{\qquad}
\date{}
%\date{October 10th, 2010}
%\date{}

%Setup hyperref package, and colours for links, text and headings

\usepackage{hyperref}           

\hypersetup{colorlinks,breaklinks, urlcolor=Maroon,linkcolor=Maroon,
  pdfauthor = {Evan K. Friis},
  pdftitle = {Statement of Research Interest},
}
%\usepackage{eurosym}
\newlength{\datebox}\settowidth{\datebox}{Summer 2007}

\newcommand{\NewWorkExperience}[3]{\noindent\hangindent=2em\hangafter=0 \parbox{\datebox}{\textit{#1}}\hspace{1.5em} #2 #3%
\vspace{0.5em}}

\newcommand{\Description}[1]{\hangindent=2em\hangafter=0\noindent\raggedright\footnotesize{#1}\par\normalsize}
\newcommand{\Sep}{\vspace{2em}}
\newcommand{\TeV}{\mathrm{Te\hspace{-.08em}V}}
\frenchspacing

\begin{document}

\thispagestyle{empty}
\begin{cv}{\spacedallcaps{Statement of Research Interest}}
\Sep

%Outline:
%Past work:
  %Searches for the MSSM Higgs boson
    %Higgs boson <-- mass
    %MSSM <-- solve hierarchy problem, dark matter candidate
  %Main contributions are novel algorithms
    %Tau ID
    %SV fit
  %Developed reconstruction techniques and software to measure backgrounds.
  %Set worlds best limits on MSSM in 2010
  %Expect even better results in 2011

%Future work:
  %Interested in working on more hardware development
  %Would like to be involved with the development of an experiment

%Interested in the neutrino sector
  %Indirect measurements are amusing
  %Neutrino mass and oscillations first real deviations from SM

My graduate career research has centered on the use of tau leptons as
a probe of new physics at the Compact Muon Solenoid (CMS) experiment.  The tau
lepton is particularly useful for searching for Higgs bosons.  The Higgs boson
is a particle hypothesized to exist in the Standard Model,
 the catalyst of the mechanism that would impart mass to the
particles that form our natural world.  If it exists, the Higgs boson
``couples'' to mass, which means that it prefers to decay to pairs of the
heaviest possible particle.  As the heaviest
lepton, the tau provides one of the cleanest signatures to look for the Higgs.  The
tau is even more important in the context of the Minimal SuperSymmetric Model
(MSSM), a promising extension to the Standard Model.  In the
MSSM, the Higgs coupling to the tau is enhanced by the model parameter
$\tan\beta$; the $H\to\tau\tau$ channel is sensitive over the entire mass range
of the MSSM Higgs. 

While the tau enjoys many phenomenological advantages, in practice tau leptons
pose many experimental challenges.  First, they are hard to find.  The tau
lepton always decays before detection, and the
majority of taus decay ``hadronically.'' Hadronic decays consist of collimated
low multiplicity sprays of pions that look very similar to
the quark and gluon jets which are produced at rates many orders of magnitude
($\approx 10^8$) higher than signals of interest.  An analysis using tau leptons
therefore requires a robust tau identification strategy to avoid being
overwhelmed by misidentified quark and gluon jets.  Once found, tau leptons
present a second challenge: a tau decay always includes one or two neutrinos, so
only a fraction of the original energy of the tau is reconstructed in the
detector.  The only experimental signature of the neutrinos is an overall low
resolution imbalance in the transverse momentum of the event.  This makes
hunting for new physics bumps difficult, as the kinematic distributions of the
visible tau decay products are smeared due to the unknown neutrino momentum.

Over the course of my graduate career I have developed innovative techniques to
address both the ``hard to find'' and ``hard to measure'' tau lepton
challenges.  To find taus, I developed a new tau identification
algorithm called the ``Tau Neural Classifier''~(TaNC).  The method uses an ensemble of
neural networks and reconstructs the different hadronic decay modes of the tau.
The TaNC algorithm reduced the background misidentification
rate of the previous algorithm used in CMS analyses by a factor of five while
maintaining the same efficiency to identify real taus.  To address the ``hard to
measure'' problem, I developed the Secondary Vertex (SV) fit.  The SVfit
algorithm uses a likelihood maximization technique to reconstruct the 
momenta of neutrinos associated to tau decays in an event.  In a search for
new particles decaying to tau pairs, the use of the SVfit allows the
\emph{full} tau pair invariant mass to be reconstructed in each event. The
full tau pair mass is the most natural and sensitive observable to the
presence of a new particle.

The culmination of these developments is the recently published result from the
Compact Muon Solenoid experiment which set the world's best limits on the MSSM
using decays to tau leptons.  I was a major contributor to the analysis methods,
measurements of the backgrounds and systematic error determination in the $H \to
\mu \tau_{\mathrm{hadrons}}$ channel, in which one of the tau leptons decays to
a muon and two neutrinos, and the other tau lepton decays to hadrons and one
neutrino.  The SVfit method was used in all channels of the analysis, and
improved the limit on the MSSM $\tan\beta$ parameter by $20\%$.  I also
contributed to the related measurement of the Standard Model $Z\to\tau\tau$
cross section. The precision of the CMS $Z\to\tau\tau$ cross section measurement
rivals that of the D\O~experiment using only $1/30$th of the data.  Over the
next few months, my research plans include improving the CMS SVfit and tau
identification algorithms and updating the MSSM Higgs search using the $2011$~CMS
dataset.  Additionally, the SVfit code has been generalized to operate on an
arbitrary number of a taus and I will support other groups in applying it to
other Higgs searches.  

I enjoy creating innovative methods that make measurements thought to be
unfeasible, possible.  The improvements provided by the new tau identification
and the SVfit allowed CMS to beat the Tevatron MSSM limit with one third of the
data thought to be necessary.   In my postgraduate career, I will continue
contributing creative ideas to the scientific community.  I will increase the
impact of my contributions by obtaining leadership roles and mentoring younger
scientists, increasing the technical diversity of my work, and working on new
types of physics analyses.

As a postdoc, I intend to play a leadership role in collaborations I am a member
of.  I have become adept in the teamwork and communication skills required to do
``big science'' during my graduate career.  Over the past three years, I have
successfully coordinated the development of the CMS tau software, which was shown to
outperform the ATLAS tau ID by a factor of four at the Tau $2010$ conference.  I
understand the organizational and political elements of productive scientific
collaboration and am ready to take a leadership role.

I am interested in becoming involved with a larger cross section of experimental
activities.  In particular, I want to become involved in hardware work and the
development of new detectors. I have a solid grounding in electronics and
embedded development. I began my undergraduate career as an EE before switching
to physics, taught the UCD physics electronics lab, and enjoy electronics as a
hobby.  I plan to develop and apply these skills in solving experimental physics
problems.  I also have mechanical design and construction experience from my
undergraduate work the Princeton Plasma Lab and the Scripps Institute of
Oceanography and would like to use these abilities as well.

\textbf{[Next paragraph only for Neutrino physics]}

Scientifically, I am very interested in conducting research in neutrino and
particle astrophysics. Neutrinos are an exciting subject to work on because
neutrino oscillations and the presence of dark matter are known phenomena that
are already beyond the Standard Model.  The presence of neutrino mass on a scale
far \emph{below} the electroweak scale is a very interesting problem.  The
effect is known to be there; like Cavendish's measurement of $G$, the neutrino
mass challenges us to refine and improve the precision of our techniques to
enable measurement.  Having worked on the largest experiment in the world, I am
interested in working on smaller experiments.  The modest (compared to the LHC)
size of neutrino and atmospheric collaborations and apparatus would allow me to
contribute to the experiments in a more holistic manner.  The recent discoveries
in the neutrino sector have exposed a crack in the Standard Model. I hope to
help design precision tools so we can lever it open and peer inside.


%\textbf{[Next paragraph only for CMS physics]}

%Scientifically, I intend to continue to work on analyses searching for new
%physics at CMS\@.  There are a number of exciting searches to be done at CMS
%using taus.  The measurement of the $Z\to\tau\tau$ cross section and the MSSM
%Higgs search have demonstrated that taus are a well understood at CMS, and can
%be used with confidence.  The tau channel will play an important role in
%discovering or ruling out the Standard Model Higgs near the LEP limit, and is an
%important probe of SUSY and exotic physics. While I am interested in continuing
%my expertise with taus, I am equally interested in exploring opportunities with
%other final state particles.  With the high rate of data provided by the LHC,
%the next two years will enable many longstanding questions about
%Nature to be addressed.  I plan to help answer them. 

\end{cv}
\end{document}
