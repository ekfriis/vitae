% Adapted from http://www.cv-templates.info/2009/10/classicthesis-currvita-latex-cv-template/

\documentclass{scrartcl}        % classe article di KOMA

\makeatletter
\let\saved@bibitem\@bibitem
\makeatother

\reversemarginpar
\newcommand{\MarginDate}[1]{\marginpar{\raggedleft\itshape\small#1}}
\usepackage[LabelsAligned]{currvita}    % un buon pacchetto per CV
%\usepackage[eulermath,nochapters]{classicthesis} % stile ClassicThesis
\usepackage[nochapters]{classicthesis} % stile ClassicThesis
\usepackage{url}        % per gli indirizzi Internet
%\usepackage{cite}
\usepackage{natbib}
\usepackage{bibentry}
\renewcommand{\cvheadingfont}{\LARGE\color{Maroon}}
\renewcommand{\cvlistheadingfont}{\large}
\renewcommand{\cvlabelfont}{\qquad}
\date{February 25th, 2011}

%Setup hyperref package, and colours for links, text and headings

\usepackage{hyperref}           

\hypersetup{colorlinks,breaklinks, urlcolor=Maroon,linkcolor=Maroon,
  pdfauthor = {Evan K. Friis},
  pdftitle = {Curriculum Vitae},
}
%\usepackage{eurosym}
\newlength{\datebox}\settowidth{\datebox}{Summer 2007}

\newcommand{\NewWorkExperience}[3]{\noindent\hangindent=2em\hangafter=0 \parbox{\datebox}{\textit{#1}}\hspace{1.5em} #2 #3%
\vspace{0.5em}}

\newcommand{\Description}[1]{\hangindent=2em\hangafter=0\noindent\raggedright\footnotesize{#1}\par\normalsize}
\newcommand{\Sep}{\vspace{2em}}

\begin{document}

\thispagestyle{empty}
\begin{cv}{\spacedallcaps{Evan Klose Friis}}
%\vspace{1.5em}
%\spacedlowsmallcaps{Personal Data}
\vspace{0.5em}
\hangindent=2em
~\\
Rue de Peupliers 12\\
1205 Gen\`{e}ve\\
Switzerland\\
\href{mailto:friis@physics.ucdavis.edu}{friis@physics.ucdavis.edu} \\
+1 (530) 207-0353

\vspace{1.5em}
\noindent\spacedlowsmallcaps{Education}
\vspace{0.5em}

\NewWorkExperience{June 2011 Est.}{University of California,}{Davis}

\Description{\MarginDate{Ph.D. Experimental High Energy Physics} 
Thesis: ``Search for Higgs bosons decaying to $\tau$ leptons at $\sqrt s = 7$ TeV'' 
Member of the Compact Muon Solenoid (CMS) collaboration at CERN~(2005--Present).
\newline Advisor: Professor~John~\textsc{Conway} }

\Sep

\NewWorkExperience{June 2005}{University of California,}{San Diego}

\Description{\MarginDate{B.S. Physics} Research experience: Non-neutral positron
plasma physics}

\vspace{1.5em}
\spacedlowsmallcaps{Research Experience}
\vspace{0.5em}

\NewWorkExperience{2009-Present}{Higgs Group,}{CMS}

\Description{\MarginDate{Higgs physics} Performed analysis searching for Minimal
Super Symmetric Model (MSSM) Higgs bosons in the $H \rightarrow \tau\tau
\rightarrow \mu + \tau_{\rm{hadrons}}$ channel.  
Set new exclusion limits on the MSSM with 36~pb$^-1$ of 2010 CMS data. 
Authored the ``Secondary Vertex Fit,''
a novel algorithm for reconstructing the invisible energy in $\tau$~lepton
events that dramatically improves kinematic resolution and event acceptance.
The Secondary Vertex Fit method improves the MSSM $\tan\beta$ exclusion limit by 
20\%, pushing the CMS result past the current Tevatron limit. } \Sep 

\NewWorkExperience{2009-Present}{Electroweak Tau Group,}{CMS}

\Description{\MarginDate{$\tau$ physics} 
Measured $Z\rightarrow\tau\tau$ cross section using the 2010 CMS data set.
Developed methods and software for data--driven estimation of backgrounds from
hadronic jet mis--tags in $\tau$ lepton analyses.  } \Sep

\NewWorkExperience{2006-Present}{Tau Physics Object Group,}{CMS}

\Description{\MarginDate{$\tau$ identification} Developed the ``Tau Neural
Classifier'' tau identification algorithm, which has a jet mis--tag rate that is a
factor of five lower than the previous algorithm used in CMS.
Measured the hadronic jet mis--tag rate from first 7~TeV
CMS datasets.  Currently serve as software librarian and liaison to CMS
reconstruction software group.} \Sep

\NewWorkExperience{2010-Present}{Luminous region monitoring at}{CMS}

\Description{\MarginDate{Beamspot}Developed and supported tools to measure and
monitor the luminous collision region at the CMS experiment.} \Sep

\NewWorkExperience{2007-2008}{Pixel Detector Calibrations,}{CMS}

\Description{\MarginDate{Pixel Detector}Designed and developed software to measure the gain,
noise, and thresholds of the sixty million channels of the CMS silicon pixel detector.
Developed data formats and tools to store calibration results in the CMS
conditions database.
} 

\vspace{1.5em}
\spacedlowsmallcaps{Awards and Honors}
\vspace{0.5em}

\Description{\MarginDate{Spring 2009} Visiting Scholar, Scuola Normale Superiore, Pisa,
Italy}
\vspace{1em}
\Description{\MarginDate{2005, 2009} UC Davis Block Grant Recipient}
\vspace{1em}
\Description{\MarginDate{2006-2007, 2009} UC Davis GAANN Fellow}

\vspace{1.5em}
\spacedlowsmallcaps{Teaching}
\vspace{0.5em}

\Description{\MarginDate{January 2010} Workshop Facilitator, EJTERM 2010, Fermi
National Laboratory}
\vspace{1em}
\Description{\MarginDate{Five quarters} UC Davis Physics 116
Electronics Lab Assistant - assisted in developing new curriculum using modern
microcontrollers.}
\vspace{1em}
\Description{\MarginDate{Three quarters} UC Davis General Physics (7/9)}
\vspace{1em}

\vspace{1.5em}
\spacedlowsmallcaps{Invited Talks}
\vspace{0.5em}

\Description{\MarginDate{September 2010} ``Tau reconstruction and identification
in CMS'', 11th International Workshop on Tau Lepton Physics, Manchester, United
Kingdom}
\vspace{1em}
\Description{\MarginDate{May 2010} ``Tau ID Object Performance '', US CMS 
Collaboration Meeting, Brown University, Providence, Rhode Island}

\vspace{1.5em}
\spacedlowsmallcaps{Publications}
\vspace{0.5em}

\Description{See attached list.}

\begingroup
\makeatletter
\let\@bibitem\saved@bibitem
\bibliographystyle{amsplain} 
\nobibliography{spires_papers}
\providecommand{\bysame}{\leavevmode\hbox to3em{\hrulefill}\thinspace}
\providecommand{\MR}{\relax\ifhmode\unskip\space\fi MR }
%\Description{\bibentry{:2009ft}}
%\Description{\bibentry{:2009hw}}
%\Description{\bibentry{Chatrchyan:2009hg}}
% Just generate all the bibliography
\nocite{*}
\endgroup

\enlargethispage{\baselineskip}
\end{cv}
\end{document}
