% Adapted from http://www.cv-templates.info/2009/10/classicthesis-currvita-latex-cv-template/

\documentclass{scrartcl}        % classe article di KOMA

\makeatletter
%\let\saved@bibitem\@bibitem
\makeatother

\reversemarginpar
\newcommand{\MarginDate}[1]{\marginpar{\raggedleft\itshape\small#1}}
\usepackage[LabelsAligned]{currvita}    % un buon pacchetto per CV
%\usepackage[eulermath,nochapters]{classicthesis} % stile ClassicThesis
\usepackage[nochapters]{classicthesis} % stile ClassicThesis
\usepackage{url}        % per gli indirizzi Internet
%\usepackage{cite}
\usepackage{natbib}
\usepackage{bibentry}
\renewcommand{\cvheadingfont}{\LARGE\color{Maroon}}
\renewcommand{\cvlistheadingfont}{\large}
\renewcommand{\cvlabelfont}{\qquad}
\date{}
%\date{October 10th, 2010}
%\date{}

%Setup hyperref package, and colours for links, text and headings

\usepackage{hyperref}           

\hypersetup{colorlinks,breaklinks, urlcolor=Maroon,linkcolor=Maroon,
  pdfauthor = {Evan K. Friis},
  pdftitle = {Curriculum Vitae},
}
%\usepackage{eurosym}
\newlength{\datebox}\settowidth{\datebox}{Summer 2007}

\newcommand{\NewWorkExperience}[3]{\noindent\hangindent=2em\hangafter=0 \parbox{\datebox}{\textit{#1}}\hspace{1.1em} #2 #3%
\vspace{0.5em}}

\newcommand{\Description}[1]{\hangindent=2em\hangafter=0\noindent\raggedright\footnotesize{#1}\par\normalsize}
\newcommand{\Sep}{\vspace{2em}}
\newcommand{\TeV}{\mathrm{Te\hspace{-.08em}V}}
\frenchspacing

\begin{document}

\thispagestyle{empty}
\begin{cv}{\spacedallcaps{Evan Klose Friis}}
%\vspace{1.1em}
%\spacedlowsmallcaps{Personal Data}
\vspace{0.5em}
\hangindent=2em
~\\
Rue Virginio-Malnati 24\\
1217 Meyrin\\
Switzerland\\
\href{mailto:evan.friis@cern.ch}{evan.friis@cern.ch} \\
+1 (530) 207-0353

\vspace{1.1em}
\noindent\spacedlowsmallcaps{Academics}
\vspace{0.5em}

\NewWorkExperience{2011-Present}{Postdoctoral Scholar,}{University of Wisconsin}

\Description{\MarginDate{Postgraduate} Maintained and supported Regional Calorimeter Trigger.  Serve as analyst and editor
for the VH$\to\tau\tau$ publications.  Convener of Tau Physics Object Group.} \Sep

\NewWorkExperience{June 2011}{University of California,}{Davis}

\Description{\MarginDate{Ph.D. Experimental High Energy Physics}Thesis:
``Search for Neutral MSSM Higgs Bosons Decaying to Pairs of $\tau$ Leptons at $\sqrt s = 7~\TeV$.''
Advisor: Professor~John~Conway }

\Sep

\NewWorkExperience{June 2005}{University of California,}{San Diego}

\Description{\MarginDate{B.S. Physics}Research experience: Designed 
LabView control system for non-neutral positron plasma trap.}

\vspace{1.1em}
\spacedlowsmallcaps{Research Experience}
\vspace{0.5em}

\NewWorkExperience{2005-Present} {Compact Muon Solenoid (CMS),} {CERN}

\Description{\MarginDate{Experiment}Member of the CMS collaboration at the Large Hadron Collider.}\Sep

\NewWorkExperience{2011-Present}{Regional Calorimeter Trigger (RCT),}{CMS}

\Description{\MarginDate{Operations} Co-responsible for on-call operation of RCT, consisting of 10 racks of custom hardware in VME crates processing 16~Tbaud of input data.
Maintained an RCT uptime of 99.8\% during 2011 and 2012 data taking.  Primary developer of RCT online control and monitoring software.
Responsible for development of new algorithms to support 2014 RCT hardware upgrade.
} \Sep

\NewWorkExperience{2009-Present}{Higgs Group,}{CMS}

\Description{\MarginDate{Higgs Physics}
Major contributor to analyses searching for Higgs bosons in the H$ \to \tau\tau
\rightarrow \mu \tau_{\rm{h}}$ and W/ZH$\to \ell(\ell)\tau\tau$ channels.
Measured $Z\rightarrow\tau\tau$ cross section using the 2010 CMS data set.
Authored the ``Secondary Vertex Fit,''
a novel dynamical likelihood algorithm for reconstructing the invisible energy in $\tau$~lepton
events that improved the performance of the CMS Higgs search by 20\%.  
Editor of the HIG-12-006, HIG-12-010, HIG-12-051, and HIG-12-053 analysis publications.} \Sep 

\NewWorkExperience{2006-Present}{Tau Physics Object Group,}{CMS}

\Description{\MarginDate{Tau Identification}Developed the ``Tau Neural
Classifier'' identification algorithm using an ensemble of neural networks, which improved the performance by a factor of five 
with respect to the previous CMS algorithm.  
Served as the coordinator and primary author of the offline tau software.  
Convener of CMS Tau group for the 2013-2014 term.} \Sep

%\NewWorkExperience{2010-2011}{Luminous Region Monitoring,}{CMS}

%\Description{\MarginDate{Beamspot}Developed and supported tools to measure and
%continuously monitor the luminous collision region at the CMS experiment.} \Sep

\NewWorkExperience{2007-2008}{Pixel Detector Calibrations,}{CMS}

\Description{\MarginDate{Pixel Detector}Designed and developed software to
measure the gain, noise, and thresholds of the sixty million channels of the CMS
silicon pixel detector.  Developed tools to store calibration
results in the CMS conditions database.} 

%\NewWorkExperience{2003-2005}{}{Scripps Institute of Oceanography}

%\Description{\MarginDate{Hydraulics Lab}Provided machine shop support to oceanographic
%experiments.} \Sep

%\NewWorkExperience{Summer 2004}{}{Princeton Plasma Physics Laboratory}

%\Description{\MarginDate{Tritium Group}Developed test stand to simulate 
%the high--frequency pulsed pressure conditions found in electron-beam 
%pumped excimer lasers.} 

\vspace{1.1em}
\spacedlowsmallcaps{Leadership}
\vspace{0.5em}

\NewWorkExperience{2013-Present}{CMS Tau Physics Object Group Convener}{}

\Description{\MarginDate{}Second highest physics management level in CMS.  
Coordinate tau-related service contributions from over 30 different physicists.
Responsible for allocating manpower and approving tau-related physics results.}\Sep

\NewWorkExperience{2009-2012}{CMS Tau Offline Software Coordinator}{}

\Description{\MarginDate{}Responsible for coordinating the development, integration, and maintenance of all CMS tau software.}

\vspace{1.1em}
\spacedlowsmallcaps{Technical Skills}
\vspace{0.5em}

\Description{\MarginDate{Software} Strong C++ and Python, multivariate and statistical analysis, ROOT, GNU/Linux, RooFit, \LaTeX, cluster computing.  
Developed data analysis framework used to process petabytes of CMS data.  Contributor to open source HEP projects.}
\vspace{1em}
\Description{\MarginDate{Electronics} Analog/Digital, board repair/rework, VME, Arduino/Atmel Microcontrollers}


\vspace{1.1em}
\spacedlowsmallcaps{Awards and Honors}
\vspace{0.5em}

\Description{\MarginDate{Spring 2010}Finalist, UC Davis Big Bang! Business Plan
Competition}
\vspace{1em}
\Description{\MarginDate{Spring 2009}Visiting Scholar, Scuola Normale Superiore, Pisa,
Italy}
\vspace{1em}
\Description{\MarginDate{2005, 2009}UC Davis Block Grant Recipient}
\vspace{1em}
\Description{\MarginDate{2006-2007, 2009}UC Davis GAANN Fellow}

\vspace{1.1em}
\spacedlowsmallcaps{Teaching}
\vspace{0.5em}

\Description{\MarginDate{January 2010}Workshop Facilitator, EJTERM 2010, 
Fermilab}
\vspace{1em}
\Description{\MarginDate{Five quarters}UC Davis Physics 116
Electronics Lab Assistant - developed new curriculum using modern
microcontrollers.}
\vspace{1em}
\Description{\MarginDate{Three quarters}UC Davis General Physics (7/9)}
%\vspace{1em}

\vspace{1.1em}
\spacedlowsmallcaps{Invited Talks}
\vspace{0.5em}

\Description{\MarginDate{Nov 2012}``Search for Higgs$\to\tau\tau$ at CMS'',
LHC Physics Center Seminar, Batavia, IL}
\vspace{1em}
\Description{\MarginDate{Nov 2012}``Search for Higgs$\to\tau\tau$ at CMS'',
Chicago Workshop on LHC Physics in the Higgs Era, Chicago, IL}
\vspace{1em}
\Description{\MarginDate{June 2012}``Search for non-SM Higgs at CMS'',
Physics@LHC, Victoria, British Columbia}
\vspace{1em}
\Description{\MarginDate{July 2011}``Beyond the Standard Model Higgs at CMS'',
Higgs Hunting 2011, Orsay, France}
\vspace{1em}
\Description{\MarginDate{April 2011}``Search for Neutral MSSM Higgs Bosons Decaying to Taus'',
LPC Physics Forum, Fermilab, Illinois}
\vspace{1em}
\Description{\MarginDate{April 2011}``Tau Reconstruction at CMS'',
Tau Portal HEFTI Workshop, Davis, California}
\vspace{1em}
\Description{\MarginDate{September 2010}``Tau Reconstruction and Identification
in CMS'', 11th International Workshop on Tau Lepton Physics, Manchester, United
Kingdom}
\vspace{1em}
\Description{\MarginDate{May 2010}``Tau ID Object Performance'', US CMS 
Collaboration Meeting, Brown University, Providence, Rhode Island}

%\vspace{1.1em}
%\spacedlowsmallcaps{Publications}
%\vspace{0.5em}

%\Description{See attached list.}

%\begingroup
%\makeatletter
%\let\@bibitem\saved@bibitem
%\bibliographystyle{lucas_unsrt} 
%\nobibliography{spires_papers}
%\providecommand{\bysame}{\leavevmode\hbox to3em{\hrulefill}\thinspace}
%\providecommand{\MR}{\relax\ifhmode\unskip\space\fi MR }
%%\Description{\bibentry{:2009ft}}
%%\Description{\bibentry{:2009hw}}
%%\Description{\bibentry{Chatrchyan:2009hg}}
%% Just generate all the bibliography
%\nocite{*}
%\endgroup

\enlargethispage{\baselineskip}
\end{cv}
\end{document}
