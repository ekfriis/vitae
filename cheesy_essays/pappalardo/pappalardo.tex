\documentclass[a4paper,10pt,oneside]{article}
\usepackage[pdftex]{hyperref}	%% many PDF options can be set here
% 1 inch margins
\usepackage{fullpage}
\usepackage{fixme}
% double spaced
\renewcommand{\baselinestretch}{2}
% no page numbers
\pagestyle {empty}

\hypersetup{ pdfauthor = {Evan K. Friis},
  pdftitle = {Papallardo Fellowship application research proposal},
}

\title{Searches for Higgs bosons using $\tau$ leptons at the LHC}
\author{Evan K. Friis}
\date{}

\begin{document}
%Outline:

%Paragraph 1:
%Taus are an important probe of Higgs physics:
 %* 10 percent of higgs decays for low mass SM higgs and all of the MSSM
 %* dominant decay mode in these regions is unfeasible to to large backgrounds
 %* 65 percent of taus decay hadronically.
 %* Tau decays before first detector and always contain missing energy in the form of neutrinos.
 %* Tau physics is difficult at hadron colliders because hadronic jets can be
 %misidentified, and because the associated neutrinos are not directly detected. 
 %* We have developed new techniques that dramatically improve tau identification
 %and the reconstruction of the neutrinos.

%Paragraph 2:
%Tau ID
 %* Critical to have high signal efficiency while rejecting multi--jet event
 %events that at a much higher rate.
 %* Traditional approaches at CMS reject QCD jets by applying a geometric isolation
 %requirement.
 %* We introduced a new algorithm called the Tau Neural Classifier (TaNC).
 %* Predicated on the idea that each hadronic decay proceeds through an intermediate
 %resonance.  
 %* An ensemble of neural nets one for each decay mode to classify each decay
 %mode.
 %* At the same signal efficiency as previous algorithm, the background rate is
 %reduced by a factor of four.
 %* For a given tau signal efficiency, the algorithm gives the lowest QCD di--jet
 %mis--tag rate of all published tau identification algorithms at hadron
 %colliders worldwide.

%Paragraph 3:
%Mass reconstruction
 %* In Higgs search natural choice is to search for bumps on Mtautau background.
 %* Difficult as tau momentum not directly reconstructed.  Visible distributions
 %smeared.
 %* Traditionally the  neutrinos are reconstructed by assuming they are collinear with the
 %visible decay products.  however, this approximation fails ~55\% (undetermined or unphysical
 %solutions) and these events must be excluded final fits.
 %* We introduced a new technique called the SV fit, which uses information from
 %the missing energy measurement, tracker information to reconstruct *all*
 %information about the original tau leptons, including displaced secondary
 %information.
 %* Method has 100\% acceptance simultaneously improves the Mtautau resolution by
 %a factor of two w.r.t. the collinear approximation.

%Paragraph 4:
%Plans
 %* New techniques developed will improve the Higgs exclusion limits CMS can set.
 %* Currently working on MSSM AH -> tautau -> mu tau analysis and expect to set
 %new MSSM exclusion limits next summer.
 %* The application of the improved tau ID and SV fit will the Htt analysis in
 %any channel.  
 %* I propose to apply these advanced techniques in other channels of the MSSM
 %Higgs search, initialing the di-hadronic tau decay mode channel.  
 %* Add SM higgs search with more data
 %* I propose to continue improving the SV fit method techniques. 
 %* Polarization information LL/RR Z LR/RL Higgs
 %* Extending to secondary fit to work on events with more than two taus.  
 %* This can improve H++.  
 %* Some dark matter signatures could manifset as production of 2* H+ H+
\maketitle 

The $\tau$ lepton is one of the most important probes of Higgs physics.  Higgs
bosons decay to pairs of $\tau$ leptons 10\% of the time in the Standard Model
(SM) when $M_H < 2M_W$, and over the entire parameter space of the Minimal Super
Symmetric Model (MSSM).  The dominant decay mode of the Higgs ($H\rightarrow b
\bar b$) is infeasible due to large backgrounds.  Tau leptons decay almost
immediately after being produced.  Their decay products contain one or two
neutrinos which are not directly measured in the detector, and in 65\% of decays
the visible decay products are hadrons.  The invisible and hadronic component of
$\tau$ decays makes using them uniquely challenging at hadron colliders.  Myself
and my colleagues at UC Davis have pioneered new techniques that address these
challenges and will improve the significance of searches for Higgs bosons at the
Compact Muon Solenoid (CMS) experiment.  I propose to apply these techniques to
existing analysis channels, investigate previously unfeasible new analyses, and
continue development of the methods.

A robust algorithm for identifying hadronic $\tau$ decays is critical to obtain
high acceptance for signal events and low mis--tag rate for the multi--jet (QCD)
background which is produced at rates $O(10^6)$ time or more higher than the
signal.  Past CMS Higgs analyses have used a geometric isolation requirement to
reject QCD mis--tags.  We introduced a new algorithm called the Tau Neural
Classifier (TaNC), which is motivated by the fact that the different hadronic
decay modes proceed through difference particle resonances.  An ensemble of
neural networks (one for each possible decay mode) is used to classify $\tau$
candidates.  Tuned to give the same signal efficiency as the isolation
algorithm, the TaNC reduces the mis--tag background rate by a factor of four.
For a given signal efficiency, the TaNC has the lowest di--jet mis--tag rate of
any published hadron collider $\tau$-ID algorithm.

In searches for Higgs bosons the natural approach is to search for bumps in the
invariant mass spectrum of $\tau$ pairs ($M_{\tau\tau}$).  However, the tau
lepton can not be directly measured in the detector due to the neutrino(s) in
the decay.  In previous analyses, the neutrinos are reconstructed by assuming
they are collinear with the visible decay products.  This approximation fails
(with undetermined or unphysical solutions) for more than half of the signal
events, which must be rejected.  The events that pass have poor resolution and
non-Gaussian tails.  We developed a new technique called the Secondary Vertex
(SV) fit, which uses information from the CMS tracker, missing energy
measurement, and the kinematic distributions of tau decays.  The method is based
on the fact that $\tau$ production and decay vertices give the direction of the
$\tau$ lepton and that this information can be used to reconstruct the neutrino.
The variables reconstructed by the method completely parameterize the physics of
the tau lepton and its decay.  The method has 100\% acceptance (all solutions
are physical), and improves the $M_{\tau\tau}$ resolution by more than a factor
of two by more than a factor of two.  This improvement both enlarges \emph{and}
sharpens the Higgs ``bump'' and improves separation from the electroweak
$Z\rightarrow\tau\tau$ irreducible background.

The two new techniques described will improve the exclusion limits (or discovery
significance!) that $H\rightarrow\tau\tau$ analyses can set at CMS.  I am
currently participating in an MSSM $A/H \rightarrow \tau\tau \rightarrow \mu
\tau_{\rm h} \nu_\mu \nu_\tau \bar {\nu_\tau}$ analysis using these new
techniques and plan
to set new exclusion limits on the MSSM this summer.  We expect the limits to be
driven by statistics, and if awarded the fellowship I would continue to
participate in this analysis and improve the limit as more data is produced.
Additionally, I propose to apply the TaNC and SV fit to another channel, where
both taus decay to hadrons.  This analysis has been abandoned since 1999, when
it was determined to be very difficult due to high QCD mis--tag backgrounds.
The TaNC and SV fit methods may make this channel a feasible probe for new
physics.  The work done in these analyses will easily be translated to SM Higgs
boson searches using taus once CMS has collected sufficient data.  Additionally,
there are exciting possibilities for improving the SV fit method.  Due to the
difference in spin, the helicities of the $\tau$ leptons are completely
correlated (LL, RR) for $Z$ boson decays, and anti-correlated (RL, LR) for Higgs
boson decays. The kinematic likelihoods used in the SV fit are sensitive to the
helicity of the taus, and the method may be provide a handle to separate Higgs
from the (currently) irreducible $Z\rightarrow\tau\tau$ background, which would
improve the exclusion limits in both the MSSM and SM searches.  Another
promising extension is using the SV fit for events with more than two $\tau$
leptons.  This could open up previously infeasible channels in multi-Higgs
events (with applications to direct dark matter searches) and doubly charged
Higgs $H^{++} \rightarrow \ell^+ \ell^+ \ell^- \ell^+$ searches.  
  % old shit
  %\subsection*{Introduction} If awarded the Papallardo fellowship, I would use
  %new and novel techniques for reconstructing and identifying tau leptons to
  %search for evidence of existence of Higgs bosons at the Compact Muon Solenoid
  %(CMS) experiment at the Large Hadron Collider.  During my graduate career, I
  %(with my colleagues at UC Davis) have lead the development 

  %\subsection*{Probing the Higgs sector with $\tau$ leptons} The $\tau$ lepton is
  %one of the most important probes of Higgs physics.  In the Standard Model
  %(SM), the Higgs to $\tau$ leptons roughly 10\% of the time for
  %low Higgs mass ($M_H < 2M_W$).  The importance of the tau is even greater in
  %the Minimal Super Symmetric Model (MSSM), where the coupling of the Higgs to
  %down--type quarks and leptons is enhanced by square of the parameter
  %$\tan\beta$.  For large $\tan\beta$, the MSSM
  %Higgs bosons decay to $\tau$ leptons $O(10\%)$ of the time for all feasible
  %Higgs masses.  The dominant Higgs decay ($~90\%$) in both SM and MSSM
  %scenarios is into a $b$ quark pair.  This channel is
  %extremely experimentally difficult due to enormous backgrounds.
  %It is therefore critical to the success of the CMS physics program to search
  %for Higgs bosons using $\tau$ leptons.

  %\subsection*{Experimental challenges of $\tau$ leptons} The tau lepton is an
  %unstable particle and always decays before reaching the innermost CMS sub
  %detector.  Due to its high mass (1.78 GeV$/c^2$), the $\tau$ is unique among
  %the leptons in that it has decays containing hadrons.  Hadronic decays
  %constitute 65\% of all $\tau$ decays, and contain a tau neutrino and
  %(typically) one to three pions.  The remaining decays leptonic, $\tau^\pm
  %\rightarrow \ell \nu_\tau \bar{\nu_\ell}$, and split evenly between electrons
  %and muons.  The structure of these decays result in the two dominant
  %challenges: (1) identifying hadronic taus, and (2) reconstructing the
  %undetected particles (neutrinos) associated with the tau decays.

  %Inclusion of hadronic decays is essential to obtain a high acceptance.
  %However, hadronic jets have a similar signature to hadronic tau decays, and
  %are produced at a much higher rate ($O(10^5)$ or more) than interesting
  %electroweak and Higgs signals.  A robust $\tau$ identification algorithm is
  %needed to prevent the signal events from being overwhelmed by mis--tagged
  %multi--jet events.

  %Once a reasonably pure set of events have been selected, the natural approach
  %to use in $H \rightarrow \tau\tau$ search is to search for new bumps in the
  %invariant mass spectrum of the $\tau$ pair.  However, as the 

  \end{document}
