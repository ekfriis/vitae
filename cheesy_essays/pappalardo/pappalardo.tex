\documentclass[a4paper,11pt,oneside]{article}
\usepackage[pdftex]{hyperref}	%% many PDF options can be set here
\usepackage{fullpage}

\hypersetup{ pdfauthor = {Evan Klose Friis},
  pdftitle = {Papallardo Fellowship application research proposal},
}

\title{Searches for Higgs bosons using $\tau$ leptons at the LHC\\
Research Proposal}
\author{Evan K. Friis}
%\date{}

\begin{document}

  \maketitle

  \section{Introduction}
  If awarded the Papallardo fellowship, I would use new and novel
  techniques for reconstructing and identifying tau leptons to search for evidence of existence of Higgs
  bosons at the Compact Muon Solenoid (CMS) experiment at the Large Hadron
  Collider.  
  During my graduate career, I (with my colleagues at UC Davis) have
  lead the development 

  \section{Probing the Higgs sector with $\tau$ leptons} The $\tau$ lepton is
  one of the most important probes of Higgs physics.  In the Standard Model
  (SM), the Higgs boson decays to $\tau$ leptons roughly 10\% of the time for
  low Higgs mass ($M_H < 2M_W$).  The importance of the tau is even greater in
  the Minimal Super Symmetric Model (MSSM), where the coupling of the Higgs to
  down--type quarks and leptons is enhanced by square of the parameter
  $\tan\beta$.  For regions of parameter space with large $\tan\beta$, the MSSM
  Higgs bosons decay to $\tau$ leptons $O(10\%)$ of the time for the entire
  Higgs mass range.  In both the SM low Higgs mass region and the MSSM, the
  dominate decay mode of the Higgs is to a pair of $b$ quarks.  This channel is
  extremely experimentally difficult due to enormous hadronic jet backgrounds.

  Having established that the 

\end{document}
