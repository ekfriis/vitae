\documentclass[a4paper,11pt,oneside]{article}
\usepackage[pdftex]{hyperref}	%% many PDF options can be set here

\hypersetup{ pdfauthor = {Evan Klose Friis},
  pdftitle = {Papallardo Fellowship application},
}

\begin{document}

  \thispagestyle{empty}

  \begin{flushleft}
    \textbf{\huge On the Possibility of the Impossible}
  \end{flushleft}
  \vspace*{6mm}
  \begin{flushleft}
    \textbf{\large Joe Sample}\\[2ex]
  \end{flushleft}
  \vspace*{1mm}
  \begin{flushleft}
    \textit{Computer Science \\
      International University Bremen \\
      Campus Ring 1 \\
      28759 Bremen \\
      Germany}
  \end{flushleft}
  \vspace*{6mm}
  \begin{flushleft}
    \textit{Type: Guided Research Proposal\\
      Date: \today \\
      Supervisor: Prof.\ V.\ Confused}
  \end{flushleft}
  \vspace*{2mm}
  ~\hrule

  \section*{Executive Summary}
  
  Consider this a separate document, although it is submitted together
  with the rest. The executive summary aims at another audience than
  the rest of the proposal. It is directed at the final decision
  maker, who typically is not an expert at all in your field, but more
  a manager kind of person. Thus, don't go into any technical
  description in the executive summary, but use catch-words and bold
  statements that highlight the importance of your project.

  (target size: 15-20 lines)

  ~\hrule

  \newpage

  \section{Introduction}

  This, like the rest, addresses fellow experts from your field (but
  not from your particular topic of research). Here you should
  technically connect to the main concepts from that field and give an
  outline of your project, stating the research/engineering question
  that you want to get answered by your project.

  (target size: 1/2 to 1 page)

  \section{Statement and Motivation of Research}

  This part should make clear which question, exactly, you are
  pursuing, and why your project is relevant/interesting. This is the
  place to cite relevant literature. Where does your project extend
  the state of the art? what weaknesses in known approaches to you
  hope to overcome? If you have carried out preliminary experiments,
  describe them here.

  (target size: 1 page)

  \section{Planned Investigation}

  This is the technical core of the proposal. Here you lay out your
  plans of how you want to answer your research question specify your
  design of experiments or simulations, point out difficulties that
  you expect to encounter, etc.

  (target size: 1 - 2 pages)

  \section{Evaluation Criteria}

  This section discusses criteria which can be used to evaluate the
  research results and which can be used to related research results
  to the already known state-of-the-art.

  (target size: 1/2 page)

  \section{Timeline}

  Here you break down the research activity into smaller steps which
  will then guide you during the guided research work. Try to define
  realistic milestones.

  (target size: 1/2 page)

  \nocite{JS06}

  \bibliographystyle{unsrt}
  \bibliography{ba-sample}

\end{document}
i
